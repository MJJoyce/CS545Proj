%%
\documentclass{aiaa-tc}% insert '[draft]' option to show overfull boxes
% \documentclass{article}% insert '[draft]' option to show overfull boxes
%%

% list packages between braces
\usepackage{graphicx}
\usepackage{varioref}%  smart page, figure, table, and equation referencing
\usepackage{wrapfig}%   wrap figures/tables in text (i.e., Di Vinci style)
\usepackage{threeparttable}% tables with footnotes
\usepackage{dcolumn}%   decimal-aligned tabular math columns
\newcolumntype{d}{D{.}{.}{-1}}
\usepackage{nomencl}%   nomenclature generation via makeindex
\makeglossary
\usepackage{subfigure}% subcaptions for subfigures
\usepackage{subfigmat}% matrices of similar subfigures, aka small mulitples
\usepackage{fancyvrb}%  extended verbatim environments
\fvset{fontsize=\footnotesize,xleftmargin=2em}
\usepackage{lettrine}%  dropped capital letter at beginning of paragraph
%\usepackage[dvips]{dropping}% alternative dropped capital package
%\usepackage[colorlinks]{hyperref}%  hyperlinks [must be loaded after dropping]
\usepackage{amsmath}

%% Daoru added the following package
\usepackage{multirow}
%%

\begin{document}

\title{CSCI545 Robotics Final Project Report\\
	Robotic Motion Planning}

 \author
{		May Ang%
		\hspace{3pt},
		Thomas Collins%
		\hspace{3pt},
		Justin Garten%
		\hspace{3pt},
		and Michael Joyce%
		\\
		\normalsize\itshape
		University of Southern California, Los Angeles, CA, 90007, USA\\
}
\maketitle

\begin{abstract}
In a world gone mad, one project dared to impose its own brand of
brutal justice on a world many thought too far gone to save.
\end{abstract}

\section{Introduction}
\label{Introduction}

\subsection{Background}

Robots are no longer confined behind striped yellow lines
with flashing lights to keep soft humans away from the moving
parts. They are operate in busy hospital hallways, kitchens, and homes
with pets, children, and clutter in a constant state of motion. As
such, the ability to evaluate and interact with those dynamic environments is
increasingly essential.

There are a number of possible approaches to this, many of which are
strongly influenced by the risks of the environment, nature of the
task, and the type of information to which the robot has
access. If uncertain, should it simply stop and wait for instructions?
Should it push through, trusting others to get out of its way? An
industrial bot capable of ripping through a wall might have very
different issues than a robotic vacuum which will at most scuff a
floorboard.

\subsection{Problem Description}

Issues relating to navigation and motion run up and down the entire
robotic stack. We chose to focus on the high-level planning aspects of
the problem, particularly relating to issues involved in dynamic,
noisy environments. Our primary approach was to explore, extend, and
evaluate three broad categories of methods for navigating in a dynamic
environment.

\subsection{Robocode Environment}

As our aim was to focus in on the planning and navigation algorithms,
we chose an environment which allowed us to abstract and encapsulate many
surrounding issues. The Robocode environment is a system designed for
simulated robotic tank tournaments. It offers flexible sensor and
control models and a relatively easily customizable environment.

Pieces:
- sensor model
- motion models (regular/advanced?)
- environment (discretization?)
- Bot coding issues
  - Threading for processing

Hm, how much to leave here versus individual sections?


\subsection{Report Outline}
In the following three sections, we discuss our efforts to explore
this task using Potential Fields, Markov Decision Processes (MDPs) and
reinforcement learning, specifically Q-learning.

\section{Potential Fields}
\label{Potential Fields}
Discussion and plots...

\section{Markov Decision Processes}
\label{Markov Decision Processes}
Discussion and plots...

\section{Q-learning}
\label{Q-learning}


\section{Conclusion}
\label{Conclusion}
To sum up...

\subsection{Future work}

\section{Acknowledgments}
Our team would thank Franz Kafka, Alan Turing, and the Ghost of
Christmas Past for their unwavering support during this project. We
would also like to pour a drink for our homies in the ground.


% \bibliographystyle{aiaa}

% \bibliography{References_Database_Daoru}

\end{document}